\documentclass{article}
\usepackage{amsfonts}
\usepackage{amsmath}

\begin{document}
\title{Econ 202A Notes: Macroeconomic Theory I}
\author{John G. Friedman}
\date{Fall 2023}
\maketitle

% TODO: Go through notes and fill in assumptions / prove things and add examples
\section{Basic Neoclassical Growth Model (Cass-Koopmans)}
\paragraph{Economy consists of many identical infinitely 
lived households -- all have the same preferences and endowments.}

\paragraph{Several Interpretations}
\begin{enumerate}
    \item Representative Agent (Robinson Crusoe)
    \item Social Planner
    \item Infinitely Lived Family (dynasty)
\end{enumerate}

\paragraph{One production sector -- output produced from capital and labor. 
Can be consumed or invested. Investment becomes productive capital 
after one period.}

\[
    \begin{aligned}
        \max_{} \sum_{t=0}^\infty \beta^t u(c_t)\\
        \text{s.t.}\ \beta \in (0,1)\\
        c_t + i_t \leq y_t = f(k_t, n_t)\\
        k_{t+1} \leq (1 - \delta) k_t + i_t\\
        0 \leq n_t \leq 1\\
        k_0, k_0 > 0
    \end{aligned}
\]

\paragraph{Note that no prices are given here!}

% add meanings of variables
\subsection{Production Function}
$F : \mathbb{R}^2_+ \rightarrow \mathbb{R}_+$
\begin{enumerate}
    \item continuously differentiable
    \item homogenous of degree 1 $\Leftrightarrow$ constant returns to scale
    \item strictly quasi-concave
    % should I also add F(k,0)?
    \item $F(0,n) = 0$
    \begin {enumerate}
        \item implies capital is essential
        \item $F_k = \text{marginal product of capital} > 0$
        \item $F_n = \text{marginal product of labor} > 0$
    \end{enumerate}
    \item $\lim_{k \to 0} F(k,1) = \infty$
    \item $\lim_{k \to \infty} F(k,1) = 0$
\end{enumerate}

\subsection{Utility Function}
$u : \mathbb{R}_+ \rightarrow \mathbb{R}$
\begin{enumerate}
    \item bounded - needed for dynamic programming
    \item continuously differentiable
    \item strictly increasing
    \item strictly concave
    \item $\lim_{c \to 0} u'(c) = \infty$
\end{enumerate}

\paragraph{Note: We will use functional forms for $F$ and $N$ for most everything we do in this class.}

% add step by step proof
\subsection{Simplifying the Planner's Problem}
\begin{enumerate}
    % add intuition
    \item $F_N > 0\ \text{and}\ u'(c) > 0 \Rightarrow n_t = 1\ \forall \ t$
    % need to look at this / clarify this
    \item $ u'(c) > 0 \Rightarrow \ \text{resource constraint holds with equality -}\ c_t + i_t = F(k_t,n_t)$
    % add intuition
    \item $ \beta < 1 \Rightarrow$ require positive rate of return to give up a unit of consumption today for 
    consumption tomorrow [actually  $mp_{k+1}-\delta$ ]
    \item Let $f(k) = F(k,1) + (1-\delta)k$
\end{enumerate}

% TODO ask someone what the bad handwriting means
\paragraph{Problem becomes:}
\[
    \begin{aligned}
        \max_{} \sum_{t=0}^\infty \beta^t u(f(k_t) - k_{t+1})\\
        \text{s.t.}\ 
        0 \leq k_{t+1} \leq f(k_t)\\
        k_0 \text{ given}
    \end{aligned}
\]

\begin{enumerate}
    \item Called a sequences problem by Stokey and Lucas
    \item Infinite number of choice variables
\end{enumerate}

\paragraph{$\Rightarrow$ Use recursive methods.}

\section{Dynamic Programming}




\end{document}