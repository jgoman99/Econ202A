\documentclass{article}
\usepackage{amsfonts}
\usepackage{amsmath}
\usepackage{xcolor}

% use newcommand to make a paragraph with a red title
% fix
\newcommand{\mynotes}[2]{\textcolor{red}{\underline{\textbf{#1}}} \paragraph{\textcolor{red}{#2}}}


\begin{document}
\title{Econ 202A Notes: Macroeconomic Theory I}
\author{John G. Friedman}
\date{Fall 2023}
\maketitle


\section{Basic Neoclassical Growth Model (Cass-Koopmans)}
\paragraph{Economy consists of many identical infinitely 
lived households -- all have the same preferences and endowments.}

\paragraph{Several Interpretations}
\begin{enumerate}
    \item Representative Agent (Robinson Crusoe)
    \item Social Planner
    \item Infinitely Lived Family (dynasty)
\end{enumerate}

\paragraph{One production sector -- output produced from capital and labor. 
Can be consumed or invested. Investment becomes productive capital 
after one period.}

\[
    \begin{aligned}
        \max_{} \sum_{t=0}^\infty \beta^t u(c_t)\\
        \text{s.t.}\ \beta \in (0,1)\\
        c_t + i_t \leq y_t = f(k_t, n_t)\\
        k_{t+1} \leq (1 - \delta) k_t + i_t\\
        0 \leq n_t \leq 1\\
        k_0, k_0 > 0
    \end{aligned}
\]

\paragraph{Note that no prices are given here!}
% not in lecture notes
{
\textcolor{red}{\paragraph{Variable Definitions}{
\begin {itemize}
    \item $\beta$ is the amount you value future consumption. Note 
that we assume you value future consumption less than present 
consumption, but more than zero.
    \item $c_t$ is consumption at time $t$
    \item $i_t$ is investment at time $t$
    \item $y_t$ is output at time $t$
    \item $k_t$ is capital at time $t$
    \item $n_t$ is labor at time $t$
    \item $\delta$ is the depreciation rate of capital
    \item $f$ is the production function
    \item $u$ is the utility function
\end{itemize}
}
}
}
\subsection{Production Function}
$F : \mathbb{R}^2_+ \rightarrow \mathbb{R}_+$
\begin{enumerate}
    \item continuously differentiable
    \item homogenous of degree 1 $\Leftrightarrow$ constant returns to scale
    \item strictly quasi-concave
    % should I also add F(k,0)?
    \item $F(0,n) = 0$
    \begin {enumerate}
        \item implies capital is essential
        \item $F_k = \text{marginal product of capital} > 0$
        \item $F_n = \text{marginal product of labor} > 0$
    \end{enumerate}
    \item $\lim_{k \to 0} F(k,1) = \infty$
    \item $\lim_{k \to \infty} F(k,1) = 0$
\end{enumerate}

\subsection{Utility Function}
$u : \mathbb{R}_+ \rightarrow \mathbb{R}$
\begin{enumerate}
    \item bounded - needed for dynamic programming
    \item continuously differentiable
    \item strictly increasing
    \item strictly concave
    \item $\lim_{c \to 0} u'(c) = \infty$
\end{enumerate}

\paragraph{Note: We will use functional forms for $F$ and $N$ for most everything we do in this class.}

% add step by step proof
\subsection{Simplifying the Planner's Problem}
\begin{enumerate}
    % add intuition
    \item $F_n > 0\ \text{and}\ u'(c) > 0 \Rightarrow n_t = 1\ \forall \ t$
    % need to look at this / clarify this
    \item $ u'(c) > 0 \Rightarrow \ \text{resource constraint holds with equality:}\ c_t + i_t = F(k_t,n_t)$
    % add intuition
    \item $ \beta < 1 \Rightarrow$ require positive rate of return to give up a unit of consumption today for 
    consumption tomorrow [actually  $mp_{k+1}-\delta$ ]
    \item Let $f(k) = F(k,1) + (1-\delta)k$
\end{enumerate}

% TODO ask someone what the bad handwriting means
\paragraph{Problem becomes:}
\[
    \begin{aligned}
        \max_{} \sum_{t=0}^\infty \beta^t u(f(k_t) - k_{t+1})\\
        \text{s.t.}\ 
        0 \leq k_{t+1} \leq f(k_t)\\
        k_0 \text{ given}
    \end{aligned}
\]

\begin{enumerate}
    \item Called a sequences problem by Stokey and Lucas
    \item Infinite number of choice variables
\end{enumerate}

\paragraph{$\Rightarrow$ Use recursive methods.}
\paragraph{Remember $f(k) = F(k,1) + (1-\delta)k$}

\section{Dynamic Programming}

\[
    \begin{aligned}
        V(k_0) = \max_{\{k_t\}_{t=0}^\infty} \sum_{t=0}^\infty \beta^t u(f(k_t) - k_{t+1})\\
        \text{given } k_0 \\
        \rightarrow \text {maximized discounted utility given } k_0
    \end{aligned}
    \]

\[
    \begin{aligned}
        V(k_0) = \max_{\{k_t\}_{t=0}^\infty} u(f(k_0)-k_1) + \beta \sum_{t=1}^\infty \beta^{t-1} u(f(k_t) - k_{t+1})\\
        = \max_{k_1} u(f(k_0)-k_1) + \beta V(k_1)\\
        \text{where } V(k_1) = \max_{\{k_t+1\}_{t=1}^\infty} \sum_{t=1}^\infty \beta^{t-1} u(f(k_t) - k_{t+1})
    \end{aligned}
    \]

$V(k) = max_{k'}\ u(f(k) - k') + \beta V(k')$
\begin{enumerate}
    \item called "Bellman's Equation"
    \item a functional equation where the unknown is $V(k)$
    \item $V(k)$ is called the value function
    \item $u(f(k) - k')$ is called the return function
\end{enumerate}

\paragraph{First order conditions \\
$u'(f(k) - k') = \beta V'(k')$ \\
$\Rightarrow$ solve for $k' = g(k)$ - The ``policy function''}

\paragraph{Solving for $V(k)$}
\begin {enumerate}
\item Guess a function $V_0(k)$
\item $T(v_0)(k) = \max_{k'} u(f(k) - k') + \beta V_0(k')$
\item Let $v_1(k) = T(v_0)(k)$
\item Repeat forming a sequence of functions where $v_n(k) = T(v_{n-1})(k)$
\end{enumerate}

% missing this from lecture slides
${V_n}$ FIX

$ V(k) = max_{k'} u(f(k) - k') + \beta V(k')$

\paragraph{Solve by computing a sequence of functions beginning with
an arbitrary $V_0(k)$\\
$V_{n+1}(k) = max_{k'} u(f(k) - k') + \beta V_n(k')$\\
That is, $V_{n+1}(k) = T(V_n)(k)$\\
We want to find fixed point: Some $V(k)$ where $V(k) = T(V(k))$}

\paragraph{Very few cases exist where a fixed point can be found analytically}
\begin{enumerate}
    \item Problem 1 on Assignment 1
    \item Return function is quadratic and constraints are linear
\end{enumerate}
\paragraph{If a closed form can be found, we can find it using the "method of undetermined coefficients"}
\paragraph{This involves two steps:}
\begin{enumerate}
    \item Find the function form for $v$
    \item Find the parameters of the fixed point in the Bellman mapping.
\end{enumerate}


% I don't get why the derivative of utility is beta valuation derivative
% what is k'

\begin{enumerate}
    % TODO is tilde over k here?
    \item Suppose $V_n(k) = ak^2 + bk + c$ and $T(v_n(k)) = \tilde{a} \tilde{k} \tilde{b}k + \tilde{c}$
    \item $T$ transforms a quadratic function into another quadratic function
    \item Fixed point is a quadratic function
    \item How do we find the fixed point?
    \item Must be the case that: $\begin{aligned}[t] \\ \tilde{a} = f_1(a,b,c) \\
    \tilde{b} = f_2(a,b,c) \\
    \tilde{c} = f_3(a, b, c)
    \end{aligned}$
    \item Hence, fixed point is a solution to a system of three equations in three unknowns: $\begin{aligned}[t] \\
    a = f_1(a,b,c) \\
    b = f_2(a,b,c) \\
    c = f_3(a,b,c)
    \end{aligned}$
    \item How do we know if V (the fixed point) exists and is unique?
    \item For review of concepts used in stating the following theoreoms, see Stokey and Lucas, Chapter 3, or
    appendix on Functional Analysis in Ljunqvist and Sargent.
\end{enumerate}

\section{Contraction mapping theorem}

\paragraph{If $(S,P)$ is a complete metric space and 
$T:S \rightarrow S$ is a 
contraction mapping with modulus $\beta$, then a $T$ has exactly one fixed point $v \in S$ 
and b for any $v_0 \in S$, $P(T^nV_0,V) \leq \beta^n P(v_0,v)$ for $n = 0,1,2,...$}

\begin{enumerate}
    \item Breaking this down:
    \item $S$ is a space of functions
    \item $S$ is a measure of distance between two points in $S$ (a metric)
        % fix this line
    \item Cauchy sequence: A sequence 
    \item Complete - Every Cauchy sequence in $S$ converges to some element of $S$
    % fix this line
    \item A "Contraction mapping" - $T : S \rightarrow S$ is a contraction 
    \item A "fixed point" is some $v \in S$ such that $T(v) = v$
    
\end{enumerate}

\paragraph{Problem: While the therorem guarantees that iterating
on Bellman's equation will provide a sequence of functions that converges
to a uniqued fixed point, verifying the conditions of the theorem is hard.}

\begin{enumerate}
    \item Blackwell's sufficient condition for a contraction:
    \item Let $X \leq R^n$ and $B(X)$ be a space of $\underline{bounded functions}$ 
    $F: X \rightarrow R$ with sup norm, $||f|| = \sup_{x \in X} |f(x)|$
    \item Let $T: B(X) \rightarrow B(X)$ satisfy
    \item a) (Monotonocity) $f,g \in B(X)$ and $f(x) \leq g(x) \forall x \in X \Rightarrow T(f)(x) \leq T(g)(x) \forall x \in X$
    \item b) (Discounting) $\exists \beta \in (0,1)$ such that $T(f+a)(x) \leq (T(f)(x))+\beta a \forall f \in B(X) and a \geq 0$
\end{enumerate}

\paragraph[exercise]{\underline{Exercise}}
{Verify $T: B(X) \rightarrow B(X)$ Monotonocity and Discounting
for the bellman mapping from the Neoclassical Growth model.}

\section{Envelope condition and Euler equation}

\begin{enumerate}
    \item First order condition for the right hande side of Bellman equation:
    \item $u'(f(k) - g(k)) = \beta V'(g(k))$ where $k' = g(k)$
    \item Envelope Theorem:
    \item Suppose V is concave and W is concave and differentiable with
    $W(X_0) = V(X_0)$ and $W(X) \leq V(X) \forall x$ in a neighborhood of $x_0$
    Then V is differentiable at $X_0$ and $V_i(X_0) = W_i(X_0)$

\end{enumerate}

%TODO add graph here

\begin{enumerate}
    \item In Bellman case, let $W(k) \equiv u(f(k) - g(k_0)) + \beta V(g(k_0))$
    \item $\Rightarrow W(k_0) = V(k_0)$ and $W(k) \leq V(k)$ for $k$ near $k_0$
    \item Envelope Conditions:
    \item $V'(k_0) = w'(k_0) = u'(f(k_0) - g(k_0))(f'(k_0))$
    %TODO fix missing
    \item in general, $V'(k) = u'(f(k) - g(k))(f'(k))$
    \item $u'(f(k) - g(k)) = \beta u'(f(g(k)) - g(g(k)))f'(g(k))$
    \item or $u'(f(k_t)-k_{t+1}) = \beta u'(f(k_{t+1}) - k_{t+2})f'(k_{t+1})$
    \item $\rightarrow$ Euler Equation, one for every $t \geq 0$
\end{enumerate}

\section{transversatality condition}
$\lim_{t \to \infty} \beta^t u'(f(k_t) - k_{t+1})f'(k_t)*k_t = 0$
See Thm 4.15 on page 98 of Stokey and Lucas

Given $k_0$ only one $k$, consistent with transversatality condition

\paragraph{Steady State}
{The steady state stock of capital is some $\bar{k}>0$} such that $\bar{k} = g(\bar{k})$
\\
\\
This can be easily found from the Euler Equation:
\\
\[
    \begin{aligned}
    u'(f(k_t) - k_{t+1}) = \beta u'(f(k_{t+1}) - k_{t+2})f'(k_{t+1})\\
    \text{by setting } k_t = k_{t+1} = k_{t+2} = k\\
    \Rightarrow u'(f(\bar{k})-\bar{k}) = \beta u'(f(\bar{k})-\bar{k})f'( \bar{k} )\\
    \Rightarrow \frac{1}{\beta} = f'( \bar{k} ) \text{, where } f(k) \equiv F(k,1) + (1-\delta)k
    \end{aligned}
\]
\\
Does a solution exist?
\\
\[
    \begin{aligned}
    \lim_{k \to 0} f'(k) = \infty \\
    \lim_{k \to \infty} f'(k) = 1 - \delta < 1 \\
    f(k) \text{ is continuously differentiable}
    \end{aligned}
\]
% TODO Add graph

\[
    \begin{aligned}
    \text{Is } k>0 \text{ unique?} \\
    f(k) \text{ is concave - See Exercise 4.8 of Stokey and Lucas. Follows 
    from quasiconcavity and CRS} \\ % What is CRS?
    \Rightarrow f''(k) < 0 \\
    \Rightarrow \bar{k} \text{ unique} \\ 
    \text{But, } \bar{k} = 0 \text{ is also a steady state since } f(0) = 0
    \end{aligned}    
\]


\paragraph[definition]{Maximum Sustainable Capital Stock}
{$k^*$, the maximum sustainable capital stock is the largest
value of $k$ such that $f(k)\geq k$}

% TODO add graph
\begin{enumerate}
    \item for k to \underline{ever} be above $k^*, k_0 > k^*$ (perhaps due to shift in production function)
    \item If so, $k$ must decrease over time until it is less than or equal to $k^*$
    \item can effectively ignore capital stocks above $k^*$
    \item can ignore consumption above $\bar{c} = f(k^*)$
    \item can bound utility by $u(\bar{c})$
    \item can apply contraction mapping theorem
\end{enumerate}

\paragraph[definition]{Dynamics of Growth Model}
{How $k_t$ evolves from $k_0>0$ depends on the policy function, $g(k)$. What do we know about this function:
\begin{enumerate}
    \item $g(0) = 0$ and increasing
    \item Must cross 45 degree line only once for $k>0$ since $\bar{k}$ is unique.
\end{enumerate}
The key to dynamics is whether or not $g(k)>k$ for arbitrarily small $k>0$
}


% todo add graph

$\Rightarrow lim_{t \to \infty} k_t = 0$ if $k_0 < \bar{k}$
$\Rightarrow$ this contradicts optimization
Hence, $g(k) > k$ for arbitrarily small $k$

% TODO add graph

\paragraph[definition]{Adding Labor-Leisure Tradeoff}
{
$\bullet$ key to real business cycle literature
\[
    \begin{aligned}
    \max_{} \sum_{t=0}^\infty \beta^t u(c_t, 1-n_t)\\
    \text{s.t.}\ c_t + k_{t+1} = F(k_t,n_t) + (1-\delta)k_t\\
    k_0 \text{ given}\\
    \end{aligned}
\]
Can make assumptions about $u$ to guarantee an interior solution where:\\
$n \in (0,1)$
}

\paragraph{Writing as a dynamic programming problem}
{
\[
    \begin{aligned}
    V(k) = \max_{k',h}{u(c,1-h)+ \beta V(k')} \\
    s.t. c+k' = F(k,h) + (1-\delta)k \\
    \text{why is $h$ not included as a state variable?} \\
    \text{because $h_0$ not needed to solve sequence problem!} \\
    \text{$h_0$ determined in period 0 and has no impact on problem solved in period 1} \\
    \text{In fact, first order condition for h (static):} \\
    u_2(c,1-h) = u_1(c, 1-h)F_2(k,h) \\
    \text{can be solved for: } h = H(k,k') \\
    \Rightarrow \text{can write Bellman equation:}
    V(k) = max_{k'}{u(c,1-H(k,k')) + \beta V(k')} \\
    s.t. c+k' = F(k,H(k,k')) + (1-\delta)k \\
    \rightarrow \text{Fits into structure of Stokey \& Lucas!}
    \end{aligned}
\]
}


\end{document}